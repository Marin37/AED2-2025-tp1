\documentclass[10pt,a4paper]{article}

\input{AEDmacros}
\usepackage{caratula} % Version modificada para usar las macros de algo1 de ~> https://github.com/bcardiff/dc-tex


\titulo{Trabajo Pr\'actico 1:}
\subtitulo{Especificación de Tads}

\fecha{\today}

\materia{Algoritmos y estructura de datos II}
\grupo{{\fontsize{14pt}{12pt}\selectfont 1er cuatrimestre de 2025
\\ \\ SyntaxError}} % 
\integrante{Apellido,Nombre}{123}{hola02@gmail.com}
\integrante{Apellido,Nombre}{123}{hola02@gmail.com}
\integrante{Abbate, Lucas}{134/21}{abbatelucas@gmail.com}
\integrante{Herrera, Alison Yamila}{814/23}{yaliherrera02@gmail.com}

% Pongan cuantos integrantes quieran

% Declaramos donde van a estar las figuras
% No es obligatorio, pero suele ser comodo
\graphicspath{{../static/}}

\begin{document}

\maketitle

\section{Resolución TP}

\begin{TAD}{Transaccion \{ } 
	\obs{id\_transaccion: \nat}
	\obs{id\_comprador: \ent}
	\obs{id\_vendedor: \nat}
	\obs{monto: \nat}
\end{TAD}
\}

\begin{TAD}{Bloque \{ }
	\obs{transacciones: \TLista{Transaccion}}
	\obs{n\_bloque: \nat}
\end{TAD}
\}

\begin{TAD}{BlockChain \{ }
	\obs{bloques: \TLista{Bloque}}
	    
\end{TAD}
\}

\begin{TAD}{\$BerretaCoin \{ }
	\obsindent{blockchain: BlockChain} % Si usas obsindent queda todo corrido a la misma altura, así que usar sólo en el ultimo
\end{TAD}
\vspace{1ex}

\begin{aux}{ultimoBloque}{\In blockchain: Blockhain}{Bloque}{
		res.n\_bloque = \longitud{blockchain} - 1
	}
\end{aux}

\begin{aux}{primeraTransaccion}{\In bloque: Bloque}{Transaccion}{
		(\longitud{bloque.transacciones} > 0) \implicaLuego bloque.transacciones[0]
	}
\end{aux}


\vspace{1ex}

\begin{proc}{agregarBloque}{\Inout blockchain: BlockChain}{}{
		\requiere{blockchain = BC_0}
		\asegura{\longitud{blockchain.bloques} > \longitud{BC_0.bloques}}
		% si ultimoBloque es menor a 3000, hay transaccion de creacion
		\asegura{(\text{ultimoBloque}(BC_0) < 3000 )\implicaLuego \newline((\text{primeraTransaccion}(blockchain)[1] = 0) \land  (\text{primeraTransaccion}(blockchain)[3] = 1))}
	}
\end{proc}
\vspace{1ex}

\begin{proc}{agregarTransaccion}{\In id\_comprador : \ent, \In id\_vendedor : \nat, \In monto : \nat, \Inout blockchain: BlockChain} {
	\requiere{id\_comprador \neq id\_vendedor}
	\asegura{\text{transaccionUnica}(res[0], blockchain)}
	}
\end{proc}



\pred{transaccionUnica}{id\_transaccion: \nat, blockchain: BlockChain}{\paraTodo[unalinea]{i}{\ent}{\paraTodo[unalinea]{j}{\ent}{(0 \le i \le \longitud{blockchain.bloques}) \yLuego (0 \le j \le \longitud{blockchain.bloques[i]}) \implicaLuego id\_transaccion \neq blockchain.bloques[i][j][0]}}} % Quizas se pueda simplificar



\end{document}